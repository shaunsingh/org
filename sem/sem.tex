% Created 2021-09-08 Wed 15:41
% Intended LaTeX compiler: pdflatex
\documentclass{scrartcl}
\usepackage[utf8]{inputenc}
\usepackage[T1]{fontenc}
\usepackage{fontspec}
\usepackage{graphicx}
\usepackage{grffile}
\usepackage{longtable}
\usepackage{wrapfig}
\usepackage{rotating}
\usepackage[normalem]{ulem}
\usepackage{amsmath}
\usepackage{textcomp}
\usepackage{amssymb}
\usepackage{capt-of}
\usepackage[dvipsnames]{xcolor}
\usepackage[colorlinks=true, linkcolor=Blue, citecolor=BrickRed, urlcolor=PineGreen]{hyperref}
\usepackage{indentfirst}
\setmainfont[Ligatures=TeX]{Alegreya}
\setmonofont[Ligatures=TeX]{Liga SFMono Nerd Font}
% features: (acronym par-sep)
\newcommand{\acr}[1]{\protect\textls*[110]{\scshape #1}}
\newcommand{\acrs}{\protect\scalebox{.91}[.84]{\hspace{0.15ex}s}}
\setlength{\parskip}{\baselineskip}
\setlength{\parindent}{0pt}

% end features

%% make document follow Emacs theme

\definecolor{obg}{HTML}{FDF6E3}
\definecolor{ofg}{HTML}{556b72}

\pagecolor{obg}
\color{ofg}

% list labels

\definecolor{itemlabel}{HTML}{268bd2}

\renewcommand{\labelitemi}{\textcolor{itemlabel}{\textbullet}}
\renewcommand{\labelitemii}{\textcolor{itemlabel}{\normalfont\bfseries \textendash}}
\renewcommand{\labelitemiii}{\textcolor{itemlabel}{\textasteriskcentered}}
\renewcommand{\labelitemiv}{\textcolor{itemlabel}{\textperiodcentered}}

\renewcommand{\labelenumi}{\textcolor{itemlabel}{\theenumi.}}
\renewcommand{\labelenumii}{\textcolor{itemlabel}{(\theenumii)}}
\renewcommand{\labelenumiii}{\textcolor{itemlabel}{\theenumiii.}}
\renewcommand{\labelenumiv}{\textcolor{itemlabel}{\theenumiv.}}

% structural elements

\definecolor{documentTitle}{HTML}{d33682}
\definecolor{documentInfo}{HTML}{d33682}
\definecolor{level1}{HTML}{268bd2}
\definecolor{level2}{HTML}{d33682}
\definecolor{level3}{HTML}{6c71c4}
\definecolor{level4}{HTML}{5ca8dd}
\definecolor{level5}{HTML}{de68a1}
\definecolor{level6}{HTML}{92c4e8}
\definecolor{level7}{HTML}{e99ac0}
\definecolor{level8}{HTML}{d3e7f6}

\addtokomafont{title}{\color{documentTitle}}
\addtokomafont{author}{\color{documentInfo}}
\addtokomafont{date}{\color{documentInfo}}
\addtokomafont{section}{\color{level1}}
\newkomafont{sectionprefix}{\color{level1}}
\addtokomafont{subsection}{\color{level2}}
\newkomafont{subsectionprefix}{\color{level2}}
\addtokomafont{subsubsection}{\color{level3}}
\newkomafont{subsubsectionprefix}{\color{level3}}
\addtokomafont{paragraph}{\color{level4}}
\newkomafont{paragraphprefix}{\color{level4}}
\addtokomafont{subparagraph}{\color{level5}}
\newkomafont{subparagraphprefix}{\color{level5}}

% textual elements

\definecolor{link}{HTML}{268bd2}
\definecolor{cite}{HTML}{800080}
\definecolor{itemlabel}{HTML}{268bd2}
\definecolor{code}{HTML}{cb4b16}
\definecolor{verbatim}{HTML}{859900}

\renewcommand{\labelitemi}{\textcolor{itemlabel}{\textbullet}}
\renewcommand{\labelitemii}{\textcolor{itemlabel}{\normalfont\bfseries \textendash}}
\renewcommand{\labelitemiii}{\textcolor{itemlabel}{\textasteriskcentered}}
\renewcommand{\labelitemiv}{\textcolor{itemlabel}{\textperiodcentered}}

\renewcommand{\labelenumi}{\textcolor{itemlabel}{\theenumi.}}
\renewcommand{\labelenumii}{\textcolor{itemlabel}{(\theenumii)}}
\renewcommand{\labelenumiii}{\textcolor{itemlabel}{\theenumiii.}}
\renewcommand{\labelenumiv}{\textcolor{itemlabel}{\theenumiv.}}

\DeclareTextFontCommand{\texttt}{\color{code}\ttfamily}
\makeatletter
\def\verbatim@font{\color{verbatim}\normalfont\ttfamily}
\makeatother

% code blocks

\definecolor{codebackground}{HTML}{FDF6E3}
\colorlet{EFD}{ofg}
\definecolor{codeborder}{HTML}{f4efdd}

%% end customisations

\author{Shaurya Singh}
\date{\today}
\title{Ap Sem Summer Assignment}
\colorlet{greenyblue}{blue!70!green}
\colorlet{blueygreen}{blue!40!green}
\providecolor{link}{named}{greenyblue}
\providecolor{cite}{named}{blueygreen}
\hypersetup{
  pdfauthor={Shaurya Singh},
  pdftitle={Ap Sem Summer Assignment},
  pdfkeywords={},
  pdfsubject={},
  pdfcreator={Emacs 28.0.50 (Org mode 9.5)},
  pdflang={English},
  breaklinks=true,
  colorlinks=true,
  linkcolor=,
  urlcolor=link,
  citecolor=cite
}
\urlstyle{same}
\begin{document}

\maketitle
\setcounter{tocdepth}{2}
\tableofcontents


\section{Lens, Issues, and Articles}
\label{sec:org56ef127}
\subsection{Lens: Enviornmental}
\label{sec:orgc29e686}
\textbf{Theme:} Technology

\textbf{Issue:} How can we use technology to efficiently generate and store energy?

\textbf{Article:} \href{https://energystorage.org/why-energy-storage/technologies/}{Hydrogen Energy Storage}

\subsection{Lens: Scientific}
\label{sec:orga9eb947}
\textbf{Theme:} Technology

\textbf{Issue:} How can we use technology to filter out artificial research papers?

\textbf{Article:} \href{https://www.nature.com/articles/d41586-021-00733-5}{The fight against fake-paper factories that churn out sham science}

\subsection{Lens: Economic}
\label{sec:org000fcd6}
\textbf{Theme:} Technology

\textbf{Issue:} How can we use AI to predict demand and reduce cost of goods?

\textbf{Article:} \href{https://hbr.org/2016/11/the-simple-economics-of-machine-intelligence}{The Simple Economics of Machine Intelligence}

\subsection{Lens: Political and Historical}
\label{sec:orgdd3a9a6}
\textbf{Theme:} Technology

\textbf{Issue:} How can we use AI to analyze and predict the results of elections?

\textbf{Article:} \href{https://www.wsj.com/articles/artificial-intelligence-shows-potential-to-gauge-voter-sentiment-11604704009}{Artificial Intelligence shows Potential to Gauge Voter Sentiment}

\subsection{Lens: Artistic and Philosophical}
\label{sec:org1fff664}
\textbf{Theme:} Technology

\textbf{Issue:} How will artificially-generated art affect art prices in the future

\textbf{Article:} \href{https://www.forbes.com/sites/jessedamiani/2020/09/21/in-this-exhibition-an-ai-dreams-up-imaginary-artworks-that-artist-alexander-reben-then-creates-irl/?sh=6c0d29e732e6}{In this exhibition an ai dreams up imaginary artworks}

\subsection{Lens: Cultural and Social}
\label{sec:orgf54eac3}
\textbf{Theme:} Technology

\textbf{Issue:} What are the effects of technology on a kids social life?

\textbf{Article:} \href{https://www.linkedin.com/pulse/impacts-technology-culture-tradition-social-values-ashes-niroula}{Impacts of technology on culture, tradition and social values}

\subsection{Lens: Futuristic}
\label{sec:orgb5562f7}
\textbf{Theme:} Technology

\textbf{Issue:} How will technology affect transportation in the future?

\textbf{Article:} \href{https://capgemini-engineering.com/us/en/insight/how-technologies-will-change-the-future-of-transport/}{How Technologies will change the future of transport}

\subsection{Lens: Ethical}
\label{sec:orgbd9efce}
\textbf{Theme:} Technology

\textbf{Issue:} How can we prevent unethical robotics in the future?

\textbf{Article:} \href{https://www.frontiersin.org/articles/10.3389/frobt.2017.00075/full}{A Review of Future and Ethical Perspectives of Robotics and AI}

\section{Choose one of the lenses with the best question which you have created. Read a few articles on the topic, and for two articles do the following:}
\label{sec:orgb087f77}
\subsection{Article 1: \href{https://www.nature.com/articles/d41586-021-00733-5}{The fight against fake-paper factories that churn out sham science}}
\label{sec:orgc47eba9}
\begin{enumerate}
\item Brief summary of the article
\begin{itemize}
\item Since last January, journals have retracted at least 370 papers that have
been publicly linked to paper mills. Many more retractions are expected to
follow; 15 are still under investigation. Research-integrity sleuths have
warned that some scientists buy papers from third-party firms to help their
careers. Image detectives who work under pseudonyms posted a list of more
than 400 published papers they said probably came from a paper mill in
January 2020. By March 2021, they had collectively listed more than 1,300
articles as possibly coming from paper mills. Around 26\% of the articles
that the sleuths alleged came from paper Mills have so far been retracted
or labeled with expressions of concern.
\end{itemize}

\item Authors Thesis:
\begin{itemize}
\item Research-integrity sleuths have repeatedly warned that some scientists buy
papers from third-party firms to help their careers
\end{itemize}

\item To what extent are the authors claims valid?
\begin{itemize}
\item The authors claims are valid, as the company publishing this article
(Nature), deals with peer-reviewing and publishing research papers like
these every day. The claims of the article are not only backed up by the
company's experience, but also anecdotes from other publishers and
researchers, who have been cited in the article.
\end{itemize}

\item What is one weakness with the author's claim?
\begin{itemize}
\item One weakness with the author's claim is that it does not reference
information that could be classified as coming from experts in the field.
For example, having statements from established peer-reviewers may be
beneficial.
\end{itemize}

\item Copy and paste 2 direct quotes that best represent this article .
\begin{itemize}
\item In a statement this year to Nature, Elsevier said that its journal editors
detect and prevent the publication of thousands of probable paper-mill
submissions each year, although some do get through.

\item We are one of a number of publishers to have been affected by such
activity.” Since last January, journals have retracted at least 370
papers that have been publicly linked to paper mills, an analysis by
Nature has found, and many more retractions are expected to follow.
\end{itemize}

\item Evaluate the article's effectiveness. Is it convincing?
\begin{itemize}
\item The article is convincing. The write is objective, to the point, and backs
up her writing with sources and citations. The site itself is trustworthy,
\emph{nature.com} is a host to many research papers and is experienced on the topic.
\end{itemize}

\item Do you agree with the author?
\begin{itemize}
\item I agree with the author.  Companies who churn out fake
manuscripts harm the work of other students and researchers who worked hard
to produce original work. Those who use fake research paper services may be
inexperienced and unfit for work.
\end{itemize}

\item Create an MLA works cited entry for each article:
\begin{itemize}
\item Else, Holly, and Richard Van Noorden. “The Fight against Fake-Paper
Factories That Churn out Sham Science.” Nature, vol. 591, no. 7851, Mar.
2021, pp. 516–19. www.nature.com,
\url{https://doi.org/10.1038/d41586-021-00733-5}.
\end{itemize}
\end{enumerate}

\subsection{Article 2:  \href{https://www.frontiersin.org/articles/10.3389/frobt.2017.00075/full}{A Review of Future and Ethical Perspectives of Robotics and AI}}
\label{sec:org76ed70c}
\begin{enumerate}
\item Brief summary of the article
\begin{itemize}
\item Authors and movie makers have been actively predicting how the future
would look with the appearance of advanced technology. Recently, business
leaders and academics have warned that advances in AI may have major
consequences for society. Both sides could do well to learn from each
other
\end{itemize}

\item Authors Thesis:
\begin{itemize}
\item This article reviews work considering both the future potential of
robotics and AI systems, and ethical considerations that need to be taken
in order to avoid a dystopian future.
\end{itemize}

\item To what extent are the authors claims valid?
\begin{itemize}
\item The author's claims are very valid. The author is experienced in the
subject and researches the subject at a prestigious university. The paper
is published under a well known publishing firm with a .org address. The
author also cites all his sources, which are from reputable authors and firms
\end{itemize}

\item What is one weakness with the author's claim?
\begin{itemize}
\item One weakness with the author's claim is that it does not reference
information that could be classified as coming from experts in the field.
For example, the United Nations has several policies and releases several
articles that concern the authors chosen subject. Something of a similar
nature would be ideal.
\end{itemize}

\item Copy and paste 2 direct quotes that best represent this article
\begin{itemize}
\item The technological transition from industrial robots to service robots
represents an evolution into more personalized systems with an increasing
degree of autonomy.
\item Ethical considerations should be taken into account by designers of
robotic and AI systems, and the autonomous systems themselves must also be aware of ethical implications of their actions.
\end{itemize}

\item Evaluate the article's effectiveness. Is it convincing?
\begin{itemize}
\item The article is convincing. The write is objective, to the point, and makes
sure to back up her writing with sources and citations. The site itself is
trustworthy experienced on the topic.
\end{itemize}

\item Do you agree with the author?
\begin{itemize}
\item I agree with the authors claim. The author takes into account several
ethical implications that prove the claim to be correct. Additionally, the
author's mention of transition from industrial to service robots connects
directly to the field of study. This helps support the claim made earlier.
\end{itemize}

\item Create an MLA works cited entry for each article
\begin{itemize}
\item Torresen, Jim. “A Review of Future and Ethical Perspectives of Robotics
and AI.” Frontiers in Robotics and AI, vol. 4, 2018, p. 75. Frontiers,
\url{https://doi.org/10.3389/frobt.2017.00075}.
\end{itemize}
\end{enumerate}
\end{document}
