% Created 2021-08-25 Wed 16:02
% Intended LaTeX compiler: pdflatex
\documentclass{scrartcl}
\usepackage[utf8]{inputenc}
\usepackage[T1]{fontenc}
\usepackage{fontspec}
\usepackage{graphicx}
\usepackage{grffile}
\usepackage{longtable}
\usepackage{wrapfig}
\usepackage{rotating}
\usepackage[normalem]{ulem}
\usepackage{amsmath}
\usepackage{textcomp}
\usepackage{amssymb}
\usepackage{capt-of}
\usepackage[dvipsnames]{xcolor}
\usepackage[colorlinks=true, linkcolor=Blue, citecolor=BrickRed, urlcolor=PineGreen]{hyperref}
\usepackage{indentfirst}
\setmonofont[Ligatures=TeX]{Liga SFMono Nerd Font}
\usepackage{chemfig}
\usepackage[version=4]{mhchem}
\usepackage{enumerate}

% features: (acronym underline par-sep)
\newcommand{\acr}[1]{\protect\textls*[110]{\scshape #1}}
\newcommand{\acrs}{\protect\scalebox{.91}[.84]{\hspace{0.15ex}s}}
\usepackage[normalem]{ulem}
\setlength{\parskip}{\baselineskip}
\setlength{\parindent}{0pt}

% end features

%% make document follow Emacs theme

\definecolor{obg}{HTML}{242730}
\definecolor{ofg}{HTML}{bbc2cf}

\pagecolor{obg}
\color{ofg}

% list labels

\definecolor{itemlabel}{HTML}{51afef}

\renewcommand{\labelitemi}{\textcolor{itemlabel}{\textbullet}}
\renewcommand{\labelitemii}{\textcolor{itemlabel}{\normalfont\bfseries \textendash}}
\renewcommand{\labelitemiii}{\textcolor{itemlabel}{\textasteriskcentered}}
\renewcommand{\labelitemiv}{\textcolor{itemlabel}{\textperiodcentered}}

\renewcommand{\labelenumi}{\textcolor{itemlabel}{\theenumi.}}
\renewcommand{\labelenumii}{\textcolor{itemlabel}{(\theenumii)}}
\renewcommand{\labelenumiii}{\textcolor{itemlabel}{\theenumiii.}}
\renewcommand{\labelenumiv}{\textcolor{itemlabel}{\theenumiv.}}

% structural elements

\definecolor{documentTitle}{HTML}{C57BDB}
\definecolor{documentInfo}{HTML}{C57BDB}
\definecolor{level1}{HTML}{51afef}
\definecolor{level2}{HTML}{C57BDB}
\definecolor{level3}{HTML}{a991f1}
\definecolor{level4}{HTML}{7cc3f3}
\definecolor{level5}{HTML}{d39ce3}
\definecolor{level6}{HTML}{a8d7f7}
\definecolor{level7}{HTML}{e2bded}
\definecolor{level8}{HTML}{dceffb}

\addtokomafont{title}{\color{documentTitle}}
\addtokomafont{author}{\color{documentInfo}}
\addtokomafont{date}{\color{documentInfo}}
\addtokomafont{section}{\color{level1}}
\newkomafont{sectionprefix}{\color{level1}}
\addtokomafont{subsection}{\color{level2}}
\newkomafont{subsectionprefix}{\color{level2}}
\addtokomafont{subsubsection}{\color{level3}}
\newkomafont{subsubsectionprefix}{\color{level3}}
\addtokomafont{paragraph}{\color{level4}}
\newkomafont{paragraphprefix}{\color{level4}}
\addtokomafont{subparagraph}{\color{level5}}
\newkomafont{subparagraphprefix}{\color{level5}}

% textual elements

\definecolor{link}{HTML}{51afef}
\definecolor{cite}{HTML}{800080}
\definecolor{itemlabel}{HTML}{51afef}
\definecolor{code}{HTML}{e69055}
\definecolor{verbatim}{HTML}{7bc275}

\renewcommand{\labelitemi}{\textcolor{itemlabel}{\textbullet}}
\renewcommand{\labelitemii}{\textcolor{itemlabel}{\normalfont\bfseries \textendash}}
\renewcommand{\labelitemiii}{\textcolor{itemlabel}{\textasteriskcentered}}
\renewcommand{\labelitemiv}{\textcolor{itemlabel}{\textperiodcentered}}

\renewcommand{\labelenumi}{\textcolor{itemlabel}{\theenumi.}}
\renewcommand{\labelenumii}{\textcolor{itemlabel}{(\theenumii)}}
\renewcommand{\labelenumiii}{\textcolor{itemlabel}{\theenumiii.}}
\renewcommand{\labelenumiv}{\textcolor{itemlabel}{\theenumiv.}}

\DeclareTextFontCommand{\texttt}{\color{code}\ttfamily}
\makeatletter
\def\verbatim@font{\color{verbatim}\normalfont\ttfamily}
\makeatother

% code blocks

\definecolor{codebackground}{HTML}{242730}
\colorlet{EFD}{ofg}
\definecolor{codeborder}{HTML}{2b2e37}

%% end customisations

\author{Shaurya Singh}
\date{\today}
\title{Ap Chem Summer Assignment \#3}
\colorlet{greenyblue}{blue!70!green}
\colorlet{blueygreen}{blue!40!green}
\providecolor{link}{named}{greenyblue}
\providecolor{cite}{named}{blueygreen}
\hypersetup{
  pdfauthor={Shaurya Singh},
  pdftitle={Ap Chem Summer Assignment \#3},
  pdfkeywords={},
  pdfsubject={},
  pdfcreator={Emacs 28.0.50 (Org mode 9.5)},
  pdflang={English},
  breaklinks=true,
  colorlinks=true,
  linkcolor=,
  urlcolor=link,
  citecolor=cite
}
\urlstyle{same}
\begin{document}

\maketitle

\section{The following reaction was performed, Identify element X.}
\label{sec:org3beaa0a}
\begin{align*}
  &\ce{Fe2O_3(s)+2X(s)} = \ce{2Fe(s)+X_2O_3(s)}\\
  &79.947g+2x=55.847g+50.982g\\
  &2x=106.829g-79.847g\\
  &2x=26.982g\\
\end{align*}
Since the atomic weight of \(\ce{2Fe}\) is the same as the given weight
(\(55.847g\)), the atomic weight of 2x is \(26.982g\) or Aluminium (\(\ce{Al}\))

\section{Balance the following equations}
\label{sec:orga41ebb2}
\begin{enumerate}
\item \(\ce{2AgI+Na2S} \rightarrow \ce{2Ag2S+NaI}\)
\item \(\ce{(NH4)2Cr2O7} \rightarrow \ce{Cr2O3+N2+4H2O}\)
\item \(\ce{Na3PO4+3HCl} \rightarrow \ce{3NaCl+H3PO4}\)
\item \(\ce{TiCl4+2H2O} \rightarrow \ce{TiO2+4HCl}\)
\item \(\ce{Ba3N2+6H2O} \rightarrow \ce{3Ba(OH)2+2NH3}\)
\item \(\ce{3HNO2+HNO3} \rightarrow \ce{2NO+H2O}\)
\end{enumerate}

\section{Balance the following equation:}
\label{sec:orgcf3c50c}
\(\ce{4NH4OH(aq)+KAI(SO)4\cdot12H2O}=\ce{Al(OH)3(s)+2(NH4)2Cr2O7+KOH(aq)+12H2O}\)
We can multiple \(\ce{NH4OH}\) by 4, and increase NH4 and H2O on the product
side to compensate

\section{Balance the following equation}
\label{sec:org2187d48}
\(\ce{2Fe+6HC2H3O2}=\ce{2Fe(C2H3O2)3+3H2}\)

\section{How many grams of water vapor can be generated from the combustion of 18.74 g of ethanol (C 2 H 6 O)?}
\label{sec:org6dc1d53}

\section{How many grams of potassium iodide are necessary to completely react with 20.61g of Mercury (II) chloride}
\label{sec:org4d16955}
First we balance the equation
\(\ce{HgCl2+2KL}=\ce{HgI2+2KCl}\)
Next we to find the total atomic weight.
\(200.59+2(35.45)+2(39.10+126.90)\)
Afterwards, we calculate the ratio needed
\(\frac{332}{271.49}=1.22\)
Finally we multiply
\(20.61*1.22=25.203\)

\section{How many grams of water vapor can be generated from the combustion of 18.74 g of ethanol (C 2 H 6 O)?}
\label{sec:orgbb89ebc}

\section{A reaction combines 64.81 grams of silver nitrate with 92.67 grams of potassium bromide}
\label{sec:org81cfc3b}
\begin{enumerate}
\item 72g
\item \(\ce{AgNO3}\) is the limiting reactant
\item 47.3g
\item 20.5\%
\end{enumerate}

\section{The moleculer weight of an insecticide, dibromoethane, is 187.9. Its molecular formula is \(\ce{C2H2Br2}\), What percent by mass of bromine does dibromoethane contain?}
\label{sec:org39413f0}
C = 12.011
H = 1.008
Br = 79.90

\(\ce{C2H4Br2}\)

= 24.022 + 4.032 + 159.8
= 187.9
=159.8/187.9
=.8505

= \%85.05

\section{A given sample of xenon fluoride contains molecules of a single type of \(\ce{XeFn}\), where n is some whole number. Given that \(9.03*10^{20}\)}
\label{sec:org57538a9}

moles = \(9.03*10^{20}/6.022*10^{23}\) = 1.5*10\textsuperscript{-3}
= 0.31
0.31/131+19n = 186,5 + 23.5n = 310
n = 4
therefore its \(\ce{XeF4}\)

\section{Molar mass of KCIO}
\label{sec:orga4f7c6f}

k = 39.0983
Cl = 35.45
O = 16.00

39.0983 + 35.45 + 3*16 = 122.55g
6.32/122.55 = 6.052 moles
2 mol KClO3 = 3 mol O2
2 = 3
0.052*3/2
= 0.078 mol

\section{The equation (balanced) is}
\label{sec:orga76d7fd}
\(\ce{Ca(OH)2+2HCl}=\ce{CaCl2 + 2H2O}\)

Therefore the coeffecient is 2

\section{the answer is}
\label{sec:org7ed573e}
1

\section{The answer is}
\label{sec:org41fc191}
\(\ce{2CHCl3 + 2Cl2}=\ce{2CCl4+2HCl}\)

\(\ce{CHCl3=}\) 119.378
\(\ce{CCl4=}\) 153.823

Theoretical mass = 153.823 * 0.097 = 15.336g
\% yield = 12.6/15.336 = \%82.16

\section{The answer is}
\label{sec:orgdb7dc1f}
Ch4 is the limiting reactant
8x 1 mol Ch4 / 16.04 g/mol = .499

.499 * 153.82 = 76.72g

\section{The answer is}
\label{sec:orgcff8923}
sodium carbonate + hydrohloric acid = sodium chloride + carbon doxide + water
= \(\ce{Na2CO3+HCl}+\ce{NaCl + CO2 + H2O}\)
= \(\ce{Na2CO3+2HCl}+\ce{2NaCl + 2CO2 + H2O}\)

\section{The answer is}
\label{sec:org2241998}
\begin{enumerate}
\item \(\ce{NaOH + KNO3}=\ce{NaNO3+KOH}\) = double replacement
\item \(\ce{CH4+2O2}=\ce{}\) = combustion
\item \(\ce{Fe + 3NaBr}=\ce{FaBr2+3Na}\) = single replacement
\item already balanced, double replacement
\item already balanced, double replacement
\item already balanced, synthesis
\item already balanced, decomposition
\end{enumerate}

\section{The answers are}
\label{sec:org1d35c07}
\begin{enumerate}
\item Ba(OH)2 -> BaO+H2O
\item Na2CO3 -> Na2O +CO2
\item 2LiCLI3 -> 2LiCL + 3O2
\item Al2O3 -> 2AL2 + O3
\item H2SO4 -> H2O + SO3
\end{enumerate}

\section{The answers are}
\label{sec:org82618ff}
\begin{enumerate}
\item 2Mg + O2 = 2MgO
\item N2 + 3H2 = 2NH3
\item S + O2 = SO2
\item CaO + H2O -> Ca(OH)2
\end{enumerate}

\section{The answers are}
\label{sec:org9ef7511}
\begin{enumerate}
\item 2H2O2 -> 2H2O + O2
\item Cu2+ + So42- + Ba2+ - 20H- -> Cu (OH)2 + BaSO4
\item Al+3Ag+ -> Al3+ + 3Ag
\item Cl2 + 2NaBr -> Br2 + 2NaCl
\item C2H6 + 3O2 -> CO2 + CO + 3H2O
\end{enumerate}

\section{The answers are}
\label{sec:orgd034fa3}
Part A:
\begin{enumerate}
\item Soluble
\item Insoluble
\item Insoluble
\item Insoluble
\item Soluble
\item Insoluble
\item Insoluble
\item Insoluble
\item Soluble
\end{enumerate}
10 Insoluble.
\begin{enumerate}
\item Insoluble
\item Soluble
\item Soluble
\item Soluble
\item Insoluble
\item Insoluble
\end{enumerate}

Part B:
\begin{enumerate}
\item \(\ce{AgBr(s)\ KNO3(aq)}\)
\(\ce{BaBr2(aq)\ KCl(aq)}\)
\(\ce{AlBr3(aq)\ KNO3(aq)}\)
\(\ce{K2SO4(aq)\ CuBr2(aq)}\)

\item \(\ce{Ag2CO3(s)\ KNO3(aq)}\)
\(\ce{NaCl(aq)\ KCl(aq)}\)
\(\ce{Al2(CO3)3(s)\ KNO3(aq)}\)
\(\ce{CuCO3(s)\ CuBr2(aq)}\)

\item \(\ce{Ag2S(s)\ KNO3(aq)}\)
\(\ce{CaCl(aq)\ KCl(aq)}\)
\(\ce{AlBr3(aq)\ KNO3(aq)}\)
\(\ce{K2SO4(aq)\ CuBr2(aq)}\)

\item \(\ce{AgOH(s)\ KNO3(aq)}\)
\(\ce{Ba(OH)2(aq)\ KCl(aq)}\)
\(\ce{Al(OH)3(aq)\ KNO3(aq)}\)
\(\ce{NH4(SO4)2(aq)\ CuBr2(aq)}\)
\end{enumerate}

*
\end{document}
