% Created 2021-07-14 Wed 14:32
% Intended LaTeX compiler: pdflatex
\documentclass[11pt]{article}
\usepackage[utf8]{inputenc}
\usepackage[T1]{fontenc}
\usepackage{graphicx}
\usepackage{grffile}
\usepackage{longtable}
\usepackage{wrapfig}
\usepackage{rotating}
\usepackage[normalem]{ulem}
\usepackage{amsmath}
\usepackage{textcomp}
\usepackage{amssymb}
\usepackage{capt-of}
\usepackage{hyperref}
\usepackage{chemfig}
\usepackage{mhchem}
\author{Shaurya Singh}
\date{\today}
\title{Ap Chem Summer Assignment \#1}
\hypersetup{
 pdfauthor={Shaurya Singh},
 pdftitle={Ap Chem Summer Assignment \#1},
 pdfkeywords={},
 pdfsubject={},
 pdfcreator={Emacs 28.0.50 (Org mode 9.5)}, 
 pdflang={English}}
\begin{document}

\maketitle

\section{Summer Assignment 1}
\label{sec:orgaf382f2}
\subsection{How many significant figures are there each of the following values}
\label{sec:org35d85c0}
\begin{enumerate}
\item 4 significant figures
\item 4 significant figures
\item 7 significant figures
\item 6 significant figures
\item 1 significant figures
\item 5 significant figures
\item 6 significant figures
\end{enumerate}

\subsection{Perform the indicated calculations on the following measured values, giving the final answer with the correct number of significant figures.}
\label{sec:org6f8e3ff}
\begin{enumerate}
\item \(16.81 + 3.2257 = 20.0357 \approx 20.04\)
\item \(324.6 * 815.991 = 264870.6786 \approx 264900\)
\item \(2.85 + 3.4621 + 1.3 = 7.6121 \approx 7.6\)
\item \(7.442 - 7.429 = 0.013\)
\item \(1.65 * 14 = 23.1 \approx 23\)
\item \(\frac{27}{4.148} = 6.509161 \approx 6.5\)
\item \([\frac{(3.901 - 3.887)}{3.901}] * 1.00 = [\frac{0.014}{3.901}] * 1.00 = 0.0036 * 1.00 = 0.0036\)
\item \(6.404 * 2.91 * (18.7 - 17.1) = 6.404 * 2.91 * 1.6 \approx 30\)
\end{enumerate}

\subsection{A sample of motor oil with a mass of 440 g occupies 500 mL. What is the density of the motor oil?}
\label{sec:org4ee03ed}
We have the following values:
\begin{align*}
&d = ?\\
&m = 440g\\
&v = 500mL
\end{align*}
We can utilize the formula \(d=\frac{m}{v}\) (density = mass/volume)
\begin{align*}
d&=\frac{m}{v}\\
&=\frac{440g}{500mL}\\
&=0.88\frac{g}{mL}\\
&\approx0.9\frac{g}{mL}
\end{align*}

\subsection{The density of an object is 16.3 g/mL. Its volume is 0.125 L. What is the mass of the object?}
\label{sec:org520fb6f}
We can apply vector analysis to solve for the correct units
\begin{center}
\begin{tabular}{ll}
16.3g & 1000 mL\\
\hline
1mL & 1L\\
\end{tabular}
\end{center}
We are left with
\begin{align*}
&= \frac{16.3g * 1000mL}{(mL)(L)}\\
&= \frac{16.3g * 1000\xout{mL}}{\xout{(mL)}(L)}\\
&= \frac{16300g}{(L)}\\
&\approx 16300g/L
\end{align*}
Now we have the following values:
\begin{align*}
&d = 16300g/L\\
&m = ?\\
&v = 0.125L
\end{align*}
We can plug these variables into \(d=\frac{m}{v}\) to calculate for mass
\begin{align*}
16300{g}/{L} &=\frac{m}{0.125L}\\
\end{align*}
Re-arranging the equation in terms of mass, we get the following
\begin{align*}
m &= 16300 * 0.125\ \frac{g\xout{L}}{\xout{L}}\\
&= 2037.5g\\
&\approx 2040g
\end{align*}

\subsection{A sample of uranium weighing 30.923 g was dropped in a graduated cylinder containing 22.30 mL of water. The volume of the water plus the sample was 23.90 mL. What is the density of uranium?}
\label{sec:org4e678c1}
The volume of the object is going to be the difference between the volume of the water and the volume of the water + object.
\begin{equation}
23.90mL - 22.30mL = 1.60mL
\end{equation}
Now we have the following values:
\begin{align*}
&d = ?\\
&m = 30.923g\\
&v = 1.60mL
\end{align*}
We can apply the same \(d=\frac{m}{v}\) to calculate for density
\begin{align*}
d&=\frac{m}{v} \\
            &=\frac{30.923g}{1.60mL}\\
            &=19.33\frac{g}{mL}\\
            &\approx19.3\frac{g}{mL}
\end{align*}

\subsection{How many protons, neutrons and electrons are in each of the following ions?}
\label{sec:orgeaf2006}
\begin{enumerate}
\item Protons = 26. Neutrons = 30. Electrons = 23
\item Protons = 20. Neutrons = 20. Electrons = 18
\item Protons = 9. Neutrons = 10. Electrons = 10
\item Protons = 15. Neutrons = 16. Electrons = 18
\item Protons = 53. Neutrons = 74. Electrons = 54
\item Protons = 53. Neutrons = 74. Electrons = 46
\end{enumerate}

\subsection{Given the position in the periodic table, what is the most likely oxidation state (or common ion charge) that each element will have when forming an ion?}
\label{sec:orgb86a179}
\begin{enumerate}
\item \(\ce{Cs}\) has a 1+ oxidation state
\item \(\ce{N}\) has a 3- oxidation state
\item \(\ce{Br}\) has a 1- oxidation state
\item \(\ce{K}\) has a 1+ oxidation state
\item \(\ce{Al}\) has a 3+ oxidation state
\item \(\ce{S}\) has a 2- oxidation state
\end{enumerate}

\subsection{Would you expect the following atoms to gain or lose electrons when forming an ion? If so, how many would be gained or lost?}
\label{sec:org4246ea0}
\begin{enumerate}
\item \(\ce{Be}\) is in Group 2, therefore it will lose 2 electrons
\item \(\ce{Cl}\) is in Group 17, therefore it will gain 1 electron
\item \(\ce{Al}\) is in group 13, therefore it will lose 3 electrons
\item \(\ce{O}\) is in group 16, therefore it will gain 2 electrons
\item \(\ce{F}\) is in group 17, therefore it will gain 1 electron
\item \(\ce{Li}\) is in group 1, therefore it will lose 1 electron
\end{enumerate}

\subsection{Name each of the following compounds:}
\label{sec:org1aa5924}
\begin{enumerate}
\item \(\ce{PbI2}\) is named as Lead(II) iodide
\item \(\ce{NH4Cl}\) is named as Ammonium chloride
\item \(\ce{Fe2O3}\) is named as Iron(III) oxide
\item \(\ce{LiH}\) is named as Lithium hydride
\item \(\ce{CsCl}\) is named as Caesium chloride
\item \(\ce{Cr(OH)1}\) is named as Chromium hydroxide
\item \(\ce{NaC2H2O2}\) is named as Sodium acetate
\item \(\ce{K2Cr2O7}\) is named as Potassium dichromate
\item \(\ce{Na2SO4}\) is named as Sodium sulfate
\end{enumerate}

\subsection{Which of the following particulate diagrams best shows the formation of water vapor from hydrogen gas and oxygen gas in a rigid container at 125\textdegree{} C?}
\label{sec:org4a12eb7}
The correct answer would be \textbf{C}. Both Oxygen and Hydrogen exist freely as molecules with two atoms each, which eliminates options A and B. As the chemical composition of water is \(\ce{H2O}\), there need to be twice as many hydrogen molecules as oxygen molecules, and so C is the only answer that makes sense.

\subsection{Name each of the following compounds. In addition, for the compounds in letters a-c, draw Lewis structures, predict VSEPR geometry and hybridization.}
\label{sec:org90e66ff}
\(\ce{NI3}\) is named as Nitrogen triiodide, and has the following Lewis Structure. It has a Trigonal pyramidal shape with 109.5° bond angles, and has a SP3 hybridization
\begin{align}
\chemfig{\charge{90=\:}{N}(-\charge{90=\:, 0:2pt=\:, -90=\:}{I})(-[:-90]\charge{0:2pt=\:, -90=\:, -180:2pt=\:}{I})(-[:-180]\charge{90=\:, -180:2pt=\:, -90=\:}{I})}
\end{align}
\(\ce{NH3}\) is named as Ammonia, and has the following Lewis Structure. It has a trigonal pyramid shape with 107° bond angles, and has a SP3 hybridization
\begin{align}
\chemfig{\charge{90=\:}{N}(-{H})(-[:-90]{H})(-[:-180]{H})}
\end{align}
\(\ce{CO}\) is named as Carbon monoxide, and has the following Lewis Structure. It has a linear shape with 180\textdegree{} Bond angles, and has a SP hybridization
\begin{align}
\chemfig{\charge{180=\:}{C}(~\charge{0=\:}{O})}
\end{align}
\(\ce{P4O10}\) is named as Tetraphosphorus decoxide,
\(\ce{N2O4}\) is named as Dinitrogen tetroxide,
\(\ce{PCl3}\) is named as Phosphorus trichloride

\subsection{Molecules that have geometries in one plane include which of the following? Draw the Lewis structures to prove your point}
\label{sec:orgca627a0}
The lewis structure for \(\ce{BCl3}\)
\begin{center}
\includegraphics[width=75px]{/Users/shauryasingh/Documents/notes/class/orgs/chem/images/BCL3.png}
\end{center}
The lewis structure for \(\ce{CHCl3}\) is
\begin{align}
\chemfig{{C}(-\charge{90=\:, 0:2pt=\:, -90=\:}{Cl})(-[:-90]\charge{0:2pt=\:, -90=\:, -180:2pt=\:}{Cl})(-[:-180]\charge{90=\:, -180:2pt=\:, -90=\:}{Cl})(-[:-270]{H})}
\end{align}
The lewis structure for \(\ce{NCl3}\) is
\begin{align}
\chemfig{\charge{90=\:}{N}(-\charge{90=\:, 0:2pt=\:, -90=\:}{Cl})(-[:-90]\charge{0:2pt=\:, -90=\:, -180:2pt=\:}{Cl})(-[:-180]\charge{90=\:, -180:2pt=\:, -90=\:}{Cl})}
\end{align}
Therefore, the correct option is \textbf{A}. Both options II and III are tetrahedral and trigonal pyramidal respectively. Option I (\(\ce{BCL3}\)) is the only one that has a geometry in one plane (trigonal planar)

\subsection{The electron-dot structure (Lewis structure) for which of the following molecules would have two lone pairs of electrons on the central atom? Again, draw the Lewis structures to prove your point.}
\label{sec:org7d80d91}
The lewis structure for \(\ce{H2S}\) is
\begin{center}
\includegraphics[width=75px]{/Users/shauryasingh/Documents/notes/class/orgs/chem/images/H2S.png}
\end{center}
The lewis structure for \(\ce{NH3}\) is
\begin{align}
\chemfig{\charge{90=\:}{N}(-{H})(-[:-90]{H})(-[:-180]{H})}
\end{align}
The lewis structure for \(\ce{CH4}\) is
\begin{align}
\chemfig{{C}(-{H})(-[:-90]{H})(-[:-180]{H})(-[:-270]{H})}
\end{align}
The lewis structure for \(\ce{HCN}\) is
\begin{align}
\chemfig{{H}-{C}(~\charge{0=\:}{N})}
\end{align}
The lewis structure for \(\ce{CO2}\) is
\begin{align}
\chemfig{(\charge{90=\:,-90=\:}{O})={C}=(\charge{90=\:,-90=\:}{O})}
\end{align}
Therefore, \(\ce{H2S}\)

\subsection{Draw Lewis structures for (a) C2H2, (b) H2O, (c) NH3, (d) HCl (e) CCl4}
\label{sec:orge056c97}
The lewis structure for \(\ce{C2H2}\) is
\begin{align}
\chemfig{{H}-{C}~{C}-{H}}
\end{align}
The lewis structure for \(\ce{H2O}\) is
\begin{center}
\includegraphics[width=75px]{/Users/shauryasingh/Documents/notes/class/orgs/chem/images/H2O.png}
\end{center}
The lewis structure for \(\ce{NH3}\) is
\begin{align}
\chemfig{\charge{90=\:}{N}(-{H})(-[:-90]{H})(-[:-180]{H})}
\end{align}
The lewis structure for \(\ce{HCL}\) is
\begin{align}
\chemfig{{H}(-\charge{90=\:, 0:2pt=\:, -90=\:}{Cl})}
\end{align}
The lewis structure for \(\ce{CCl4}\) is
\begin{align}
\chemfig{{C}(-\charge{90=\:, 0:2pt=\:, -90=\:}{Cl})(-[:-90]\charge{0:2pt=\:, -90=\:, -180:2pt=\:}{Cl})(-[:-180]\charge{90=\:, -180:2pt=\:, -90=\:}{Cl})(-[:-270]\charge{90=\:, 180:2pt=\:, 0=\:}{Cl})}
\end{align}

\subsection{Give the VSEPR geometry for each for each of the molecules listed in \#14.}
\label{sec:orga482973}
\begin{enumerate}
\item \(\ce{C2H2}\) has a linear VSEPR geometry
\item \(\ce{H2O}\) has a bent VSEPR geometry
\item \(\ce{NH3}\) has a trigonal pyramidal VSEPR geometry
\item \(\ce{HCl}\) has a linear VSEPR geometry
\item \(\ce{CCl4}\) has a tetrahedral VSEPR geometry
\end{enumerate}

\subsection{Tell whether each of the molecules listed in \#14 is polar or nonpolar.}
\label{sec:org1ad73db}
\begin{enumerate}
\item \(\ce{C2H2}\) is nonpolar
\item \(\ce{H2O}\) is polar
\item \(\ce{NH3}\) is polar
\item \(\ce{HCl}\) is polar
\item \(\ce{CCl4}\) is nonpolar
\end{enumerate}

\subsection{What is the type of bond for \(\ce{C2H2}\)?}
\label{sec:org4914ae3}
The lewis structure for \(\ce{C2H2}\) is,
\begin{align}
\chemfig{{H}-{C}~{C}-{H}}
\end{align}
Therefore it will have a covalent bond, and would be nonpolar.

\subsection{Name each of the following compounds:}
\label{sec:org98e935a}
\begin{enumerate}
\item \(\ce{P4O6}\) is named as Tetraphosphorus hexoxide
\item \(\ce{KOH}\) is named as Potassium hydroxide
\item \(\ce{N2}\) is named as Dinitrogen (or Nitrogen Gas)
\item \(\ce{PH3}\) is named as Monophosphorus Trihydride
\item \(\ce{BF3}\) is named as Boron trifluoride
\item \(\ce{AgCl}\) is named as Silver(I) chloride
\item \(\ce{KHCO3}\) is named as Potassium bicarbonate
\item \(\ce{AgNO3}\) is named as Silver(I) nitrate
\end{enumerate}

\subsection{Write formulas for each of the following compounds:}
\label{sec:org516c08b}
\begin{enumerate}
\item The formula for sodium cyanide is \(\ce{NaCN}\)
\item The formula for tin(II) fluoride is \(\ce{SnF2}\)
\item The formula for lead(II) nitrate is \(\ce{PbF2}\)
\item The formula for iron(III) oxide is \(\ce{FeF3}\)
\item The formula for calcium phosphate is \(\ce{Ca3(PO4)2}\)
\item The formula for sodium bromate is \(\ce{NaBrO3}\)
\item The formula for hydrogen iodide is \(\ce{HI}\)
\item The formula for sodium sulfate is \(\ce{Na2SO4}\)
\item The formula for manganese dioxide is \(\ce{MnO2}\)
\item The formula for potassium chlorate is \(\ce{KClO3}\)
\item The formula for potassium hypochlorite is \(\ce{KclO}\)
\item The formula for lithium hydride is \(\ce{LiH}\)
\item The formula for barium chloride is \(\ce{BaCl2}\)
\item The formula for magnesium oxide is \(\ce{MgO}\)
\item The formula for copper(I) oxide is \(\ce{Cu2O}\)
\end{enumerate}

\subsection{Give the names of the following acids}
\label{sec:org6c2db06}
\begin{enumerate}
\item \(\ce{H2SO3}\) is named as Sulfurous acid
\item \(\ce{HI}\) is named as Hydroiodic acid
\item \(\ce{HBr}\) is named as Hydrobromic acid
\item \(\ce{HNO2}\) is named as Nitrous acid
\item \(\ce{H3PO4}\) is named as Phosphoric Acid
\item \(\ce{HCl}\) is named as Hydrochloric acid
\end{enumerate}

\subsection{Give formulas for the following acids:}
\label{sec:orgb4358f1}
\begin{enumerate}
\item Nitric acid has a formula of \(\ce{HNO3}\)
\item hydrofluoric acid has a formula of \(\ce{HF}\)
\item sulfuric acid has a formula of \(\ce{H2SO4}\)
\item hydrocyanic acid has a formula of \(\ce{HCN}\)
\item acetic acid has a formula of \(\ce{C2H4O2}\)
\end{enumerate}

\subsection{Give the names and formulas of the seven diatomic elements.}
\label{sec:org9f361d1}
\begin{enumerate}
\item \(\ce{H2}\), or Hydrogen
\item \(\ce{N2}\), or Nitrogen
\item \(\ce{O2}\), or Oxygen
\item \(\ce{F2}\), or Fluorine
\item \(\ce{Cl2}\), or Chlorine
\item \(\ce{Br2}\), or Bromine
\item \(\ce{I2}\), or Iodine
\end{enumerate}

\subsection{Solve the following problems involving scientific notation without a calculator.}
\label{sec:orgc820d7f}
\begin{enumerate}
\item The solution is \(8*10^7\)
\begin{align*}
(2*10^3)(4*10^4)&=(2*4)(10^3*10^4)\\
&=8(10^3*10^4)\\
&=8*10^{^}{3+4}\\
&=8*10^7
\end{align*}
\item The solution is \(42*10^{11}\)
\begin{align*}
(6*10^5)(7*10^6)&=(6*7)(10^5*10^6)\\
&=42(10^5*10^6)\\
&=42*10^{^}{5+6}\\
&=42*10^{11}
\end{align*}
\item The solution is \(105*10^{12}\)
\begin{align*}
(7*10^4)(5*10^6)(3*10^2)&=(7*5*3)(10^4*10^6*10^2)\\
&=105(10^4*10^6*10^2)\\
&=105*10^{^}{4+6+2}\\
&=105*10^{12}^{}^{}
\end{align*}
\item The solution is \(2.5*10^3\)
\begin{align*}
\frac{(2*10^7)}{(8*10^3)}&=\frac{2}{8}*\frac{10^7}{10^3}\\
&=\frac{2}{8}*\frac{10^{7-3}}{1}\\
&=0.4*10^{7-3}\\
&=0.4*10^4^{}\\
&=4*10^3
\end{align*}
\item The solution is \(2*10^2\)
\begin{align*}
\frac{(4*10^6)}{(2*10^4)}&=\frac{4}{2}*\frac{10^6}{10^4}\\
&=2*10^{6-4}\\
&=2*10^2
\end{align*}
\item The solution is \(5*10^{10}\)
\begin{align*}
\frac{(2*10^3)}{(4*10^{-8})}&=\frac{2}{4}*\frac{10^3}{10^-8}\\
&=0.5*10^{3-(-8)}\\
&=0.5*10^{3+8}\\
&=0.5*10^11\\
&=5*10^{10}
\end{align*}
\item The solution is \(6*10^8\)
\begin{align*}
\frac{(5*10^6)(2*10^3)(3*10^3)}{(5*10^4)}&=\frac{(5*2*3)(10^6*10^3*10^3)}{(5*10^4)}\\
&=\frac{(30)(10^{6+3+3})}{(5*10^4)}\\
&=\frac{(30)(10^{12}^{})}{(5*10^4)}\\
&=\frac{(3*10^{13}^{}^{})}{(5*10^4)}\\
&=\frac{3}{5}*\frac{10^{13}}{10^4}\\
&=0.6*10^{13-4}\\
&=0.6*10^9\\
&=6*10^8
\end{align*}
\item The solution is \(5*10^2^{}\)
\begin{align*}
\frac{(4*10^6)(5*10^{-3})}{(8*10^{-4})(5*10^3)}&=\frac{(4*5)(10^6*10^{-3})}{(8*5)(10^{-4}*10^3)}\\
&=\frac{(20)(10^{6-3}^{})}{(40)(10^{-4+3})}\\
&=\frac{(20)(10^3^{})}{(40)(10^{-1})}^{}\\
&=\frac{(2)(10^4^{})}{(4)(10^1)}^{}\\
&=\frac{2}{4}*\frac{10^4}{10^1}\\
&=0.5*10^{4-1}\\
&=0.5*10^3\\
&=5*10^2
\end{align*}
\end{enumerate}

\subsection{What is the formula for nitric acid?}
\label{sec:org8450154}
Nitric acid has a formula of \(\ce{HNO3}\)
\end{document}
