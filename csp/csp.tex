% Created 2021-08-15 Sun 16:33
% Intended LaTeX compiler: pdflatex
\documentclass{scrartcl}
\usepackage[utf8]{inputenc}
\usepackage[T1]{fontenc}
\usepackage{fontspec}
\usepackage{graphicx}
\usepackage{grffile}
\usepackage{longtable}
\usepackage{wrapfig}
\usepackage{rotating}
\usepackage[normalem]{ulem}
\usepackage{amsmath}
\usepackage{textcomp}
\usepackage{amssymb}
\usepackage{capt-of}
\usepackage[dvipsnames]{xcolor}
\usepackage[colorlinks=true, linkcolor=Blue, citecolor=BrickRed, urlcolor=PineGreen]{hyperref}
\usepackage{indentfirst}
\setmonofont[Ligatures=TeX]{Liga SFMono Nerd Font}
% features: (acronym par-sep)
\newcommand{\acr}[1]{\protect\textls*[110]{\scshape #1}}
\newcommand{\acrs}{\protect\scalebox{.91}[.84]{\hspace{0.15ex}s}}
\setlength{\parskip}{\baselineskip}
\setlength{\parindent}{0pt}

% end features

%% make document follow Emacs theme

\definecolor{obg}{HTML}{242730}
\definecolor{ofg}{HTML}{bbc2cf}

\pagecolor{obg}
\color{ofg}

% list labels

\definecolor{itemlabel}{HTML}{51afef}

\renewcommand{\labelitemi}{\textcolor{itemlabel}{\textbullet}}
\renewcommand{\labelitemii}{\textcolor{itemlabel}{\normalfont\bfseries \textendash}}
\renewcommand{\labelitemiii}{\textcolor{itemlabel}{\textasteriskcentered}}
\renewcommand{\labelitemiv}{\textcolor{itemlabel}{\textperiodcentered}}

\renewcommand{\labelenumi}{\textcolor{itemlabel}{\theenumi.}}
\renewcommand{\labelenumii}{\textcolor{itemlabel}{(\theenumii)}}
\renewcommand{\labelenumiii}{\textcolor{itemlabel}{\theenumiii.}}
\renewcommand{\labelenumiv}{\textcolor{itemlabel}{\theenumiv.}}

% structural elements

\definecolor{documentTitle}{HTML}{C57BDB}
\definecolor{documentInfo}{HTML}{C57BDB}
\definecolor{level1}{HTML}{51afef}
\definecolor{level2}{HTML}{C57BDB}
\definecolor{level3}{HTML}{a991f1}
\definecolor{level4}{HTML}{7cc3f3}
\definecolor{level5}{HTML}{d39ce3}
\definecolor{level6}{HTML}{a8d7f7}
\definecolor{level7}{HTML}{e2bded}
\definecolor{level8}{HTML}{dceffb}

\addtokomafont{title}{\color{documentTitle}}
\addtokomafont{author}{\color{documentInfo}}
\addtokomafont{date}{\color{documentInfo}}
\addtokomafont{section}{\color{level1}}
\newkomafont{sectionprefix}{\color{level1}}
\addtokomafont{subsection}{\color{level2}}
\newkomafont{subsectionprefix}{\color{level2}}
\addtokomafont{subsubsection}{\color{level3}}
\newkomafont{subsubsectionprefix}{\color{level3}}
\addtokomafont{paragraph}{\color{level4}}
\newkomafont{paragraphprefix}{\color{level4}}
\addtokomafont{subparagraph}{\color{level5}}
\newkomafont{subparagraphprefix}{\color{level5}}

% textual elements

\definecolor{link}{HTML}{51afef}
\definecolor{cite}{HTML}{800080}
\definecolor{itemlabel}{HTML}{51afef}
\definecolor{code}{HTML}{e69055}
\definecolor{verbatim}{HTML}{7bc275}

\renewcommand{\labelitemi}{\textcolor{itemlabel}{\textbullet}}
\renewcommand{\labelitemii}{\textcolor{itemlabel}{\normalfont\bfseries \textendash}}
\renewcommand{\labelitemiii}{\textcolor{itemlabel}{\textasteriskcentered}}
\renewcommand{\labelitemiv}{\textcolor{itemlabel}{\textperiodcentered}}

\renewcommand{\labelenumi}{\textcolor{itemlabel}{\theenumi.}}
\renewcommand{\labelenumii}{\textcolor{itemlabel}{(\theenumii)}}
\renewcommand{\labelenumiii}{\textcolor{itemlabel}{\theenumiii.}}
\renewcommand{\labelenumiv}{\textcolor{itemlabel}{\theenumiv.}}

\DeclareTextFontCommand{\texttt}{\color{code}\ttfamily}
\makeatletter
\def\verbatim@font{\color{verbatim}\normalfont\ttfamily}
\makeatother

% code blocks

\definecolor{codebackground}{HTML}{242730}
\colorlet{EFD}{ofg}
\definecolor{codeborder}{HTML}{2b2e37}

%% end customisations

\author{Shaurya Singh}
\date{\today}
\title{Ap CSP Summer Assignment}
\colorlet{greenyblue}{blue!70!green}
\colorlet{blueygreen}{blue!40!green}
\providecolor{link}{named}{greenyblue}
\providecolor{cite}{named}{blueygreen}
\hypersetup{
  pdfauthor={Shaurya Singh},
  pdftitle={Ap CSP Summer Assignment},
  pdfkeywords={},
  pdfsubject={},
  pdfcreator={Emacs 28.0.50 (Org mode 9.5)},
  pdflang={English},
  breaklinks=true,
  colorlinks=true,
  linkcolor=,
  urlcolor=link,
  citecolor=cite
}
\urlstyle{same}
\begin{document}

\maketitle
\setcounter{tocdepth}{2}
\tableofcontents


\section{Digital Explosion}
\label{sec:orgf39fd3d}
\subsection{Vocabulary}
\label{sec:org4d80384}
\begin{enumerate}
\item Word: bit

Definition: (n) \textbf{bit} (a unit of measurement of information (from binary + digit); the amount of information in a system having two equiprobable states) ``there are 8 bits in a byte''

Ref: WordNet Search - 3.1. \url{http://wordnetweb.princeton.edu/perl/webwn?s=bit\&sub=Search+WordNet\&o2=\&o0=1\&o8=1\&o1=1\&o7=\&o5=\&o9=\&o6=\&o3=\&o4=\&h}=. Accessed 18 July 2021.

\item Word: koan

Definition: (n) \textbf{koan} (a paradoxical anecdote or a riddle that has no solution; used in Zen Buddhism to show the inadequacy of logical reasoning)

Ref: WordNet Search - 3.1. \url{http://wordnetweb.princeton.edu/perl/webwn?s=koan\&sub=Search+WordNet\&o2=\&o0=1\&o8=1\&o1=1\&o7=\&o5=\&o9=\&o6=\&o3=\&o4=\&h}=. Accessed 18 July 2021.

\item Word: ping

Definition: (v) \textbf{ping} (send a message from one computer to another to check whether it is reachable and active) ``ping your machine in the office''

Ref: WordNet Search - 3.1. \url{http://wordnetweb.princeton.edu/perl/webwn?s=Ping\&sub=Search+WordNet\&o2=\&o0=1\&o8=1\&o1=1\&o7=\&o5=\&o9=\&o6=\&o3=\&o4=\&h}=. Accessed 23 July 2021.

\item Word: benign

Definition: (adj) \textbf{benign} (not dangerous to health; not recurrent or progressive (especially of a tumor))

Ref: WordNet Search - 3.1. \url{http://wordnetweb.princeton.edu/perl/webwn?s=benign\&sub=Search+WordNet\&o2=\&o0=1\&o8=1\&o1=1\&o7=\&o5=\&o9=\&o6=\&o3=\&o4=\&h=0000000}. Accessed 23 July 2021.

\item Word: utopian

Definition: (n) \textbf{utopian} (an idealistic (but usually impractical) social reformer) ``a Utopian believes in the ultimate perfectibility of man''

Ref: WordNet Search - 3.1. \url{http://wordnetweb.princeton.edu/perl/webwn?s=utopian\&sub=Search+WordNet\&o2=\&o0=1\&o8=1\&o1=1\&o7=\&o5=\&o9=\&o6=\&o3=\&o4=\&h=0000000}. Accessed 23 July 2021.

\item Word: err

Definition: (v) \textbf{err}, mistake, slip (to make a mistake or be incorrect)

Ref: WordNet Search - 3.1. \url{http://wordnetweb.princeton.edu/perl/webwn?s=err\&sub=Search+WordNet\&o2=\&o0=1\&o8=1\&o1=1\&o7=\&o5=\&o9=\&o6=\&o3=\&o4=\&h=0000000}. Accessed 23 July 2021.

\item Word: paradoxically

Definition: (adv) \textbf{paradoxically} (in a paradoxical manner) ``paradoxically,  ice ages seem to occur when the sun gets hotter''

(adj) \textbf{paradoxical}, self-contradictory (seemingly contradictory but nonetheless possibly true) ``it is paradoxical that standing is more tiring than walking''

Ref: WordNet Search - 3.1. \url{http://wordnetweb.princeton.edu/perl/webwn?s=paradoxically\&sub=Search+WordNet\&o2=\&o0=1\&o8=1\&o1=1\&o7=\&o5=\&o9=\&o6=\&o3=\&o4=\&h=0000000}. Accessed 23 July 2021.

\item Word: expunge

Definition: (v)  \textbf{expunge} (remove by erasing or crossing out or as if by drawing a line)

Ref: WordNet Search - 3.1.
\url{http://wordnetweb.princeton.edu/perl/webwn?s=expunge\&sub=Search+WordNet\&o2=\&o0=1\&o8=1\&o1=1\&o7=\&o5=\&o9=\&o6=\&o3=\&o4=\&h=0000000}. Accessed 23 July 2021.

\item Word: database

Definition: (n) \textbf{database} (an organized body of related information)

Ref: WordNet Search - 3.1.
\url{http://wordnetweb.princeton.edu/perl/webwn?s=database\&sub=Search+WordNet\&o2=\&o0=1\&o8=1\&o1=1\&o7=\&o5=\&o9=\&o6=\&o3=\&o4=\&h=0000000}. Accessed 23 July 2021.

\item Word: blacklist

Definition: (v) \textbf{blacklist} (put on a blacklist so as to banish or cause to be boycotted) ``many books were blacklisted by the Nazis''

Ref: WordNet Search - 3.1.
\url{http://wordnetweb.princeton.edu/perl/webwn?s=blacklist\&sub=Search+WordNet\&o2=\&o0=1\&o8=1\&o1=1\&o7=\&o5=\&o9=\&o6=\&o3=\&o4=\&h=0000000}. Accessed 23 July 2021.
\end{enumerate}

\subsection{Ideas:}
\label{sec:org168aa52}
\begin{enumerate}
\item Companies keep records of cellphone locations (Page 1)
\item ``Its all just bits'' (Page 5)
\item ``In fact, processors have hardly grown faster at all'' (Page 8)
\item ``By 2011, we may be producing more bits than we can store'' (Page 10)
\end{enumerate}

\subsection{Journal Entry}
\label{sec:org4a9dd75}
\begin{enumerate}
\item I support the idea of companies keeping track of cellphone locations, as long
as that information is kept private and in the hands of only the government.
Cellular locations are incredibly useful for finding missing people, and
quickly reacting to emergencies, such as in Tanya's case.

However, the key term is \emph{as long as that information is kept private and in
the hands of the government}. Private companies shouldn't be able to get the
locations of users. These locations are often sold to advertising firms.
Companies like Cuebiq make money by collecting location data from smartphone
users who agree to share their locations for weather or maps, then analyse
and sell that data to advertisers and marketers. Location data shouldn't be  used to target people, and should be used as a last resort

\item While technically it is all just bits, personally I believe we shouldn't
think of it as such. Thinking of computers in terms of bits is like thinking
of writing in terms of atoms on a piece of paper. While all handwritten work
is technically just graphite on paper, we interpret it as much more than
that. We think of writing in terms of words, phrases, paragraphs, and should
think of the computer experience in terms of experiences. Similarly,
copyright law is based on text, and so laws pertaining to computers should be
based on the end user's experience, not what delivers that experience

\item Considering the next sentence is talking about ``multiple processors on the
same chip'' I assume this statement is talking about the processor cores
themselves. However, data shows that the fastest processors today are about
80 times faster in terms of single core performance with the same efficiency.
Still, its true that processor cores have increased over the years (from
single core chips to server chips with over 128 cores in the span of just 20
years).

In my opinion we should focus more on processor efficiency and less on raw
processor speed. Processor's these days, even budget ones, are more than fast
enough for the majority of use cases. The market for large, high end HEDT
processors is extremely small compared to the market share of their lower end
counterparts. The future is about switching to arm-based processors in
laptops, which should give much more performance at the same wattage.

\item I found this statement the most interesting in this chapter. We've made leaps
and bounds in storage technology since then, nowadays you can buy tens of
terabytes of storage for quite cheap, compared to the expensive ``high end''
80gb drives you could get in 2008, back when this textbook
released.

Similar my opinion on processor speed, instead of focusing on storage size
and how much data we can store, we should talk about how we store data.
Companies are moving to cloud-based centralized storage, and as of 2020 50\%
of all corporate data is stored in the cloud, up from 30\% just 5 years
earlier. As more and more people start using cloud services (e.g. OneDrive,
iCloud, Google Drive), we should focus on make data transfer to and from
those drives quicker and more secure.
\end{enumerate}

\section{Naked in the Sunlight}
\label{sec:orgba59c42}
\subsection{Vocabulary}
\label{sec:orgaffcec1}
\begin{enumerate}
\item Word: pervasive

Definition: per·va·sive | pərˈvāsiv | adjective (especially of an unwelcome influence or physical effect) spreading widely throughout an area or a group of people: ageism is pervasive and entrenched in our society.

Ref: WordNet Search - 3.1. \url{http://wordnetweb.princeton.edu/perl/webwn?s=pervasive\&sub=Search+WordNet\&o2=\&o0=1\&o8=1\&o1=1\&o7=\&o5=\&o9=\&o6=\&o3=\&o4=\&h}=. Accessed 18 July 2021.

\item Word: cleric

Definition: cler·ic | ˈklerik | noun a priest or religious leader, especially a Christian or Muslim one.

Ref: WordNet Search - 3.1. \url{http://wordnetweb.princeton.edu/perl/webwn?s=cleric\&sub=Search+WordNet\&o2=\&o0=1\&o8=1\&o1=1\&o7=\&o5=\&o9=\&o6=\&o3=\&o4=\&h}=. Accessed 18 July 2021.

\item Word: disseminate

Definition: dis·sem·i·nate | dəˈseməˌnāt | verb [with object ] spread (something, especially information) widely: health authorities should foster good practice by disseminating information.

Ref: WordNet Search - 3.1. \url{http://wordnetweb.princeton.edu/perl/webwn?s=disseminate\&sub=Search+WordNet\&o2=\&o0=1\&o8=1\&o1=1\&o7=\&o5=\&o9=\&o6=\&o3=\&o4=\&h}=. Accessed 23 July 2021.

\item Word: encode

 Definition: en·code | inˈkōd, enˈkōd | verb [with object ] convert into a
coded form: using this technique makes it possible to encode and transmit recorded video information.

Ref: WordNet Search - 3.1. \url{http://wordnetweb.princeton.edu/perl/webwn?s=encode\&sub=Search+WordNet\&o2=\&o0=1\&o8=1\&o1=1\&o7=\&o5=\&o9=\&o6=\&o3=\&o4=\&h=0000000}. Accessed 23 July 2021.

\item Word: RFID

 Definition: RFID \textbf{(abbreviation)} radio frequency identification, denoting
technologies that use radio waves to identify people or objects carrying encoded microchips.

Ref: WordNet Search - 3.1. \url{http://wordnetweb.princeton.edu/perl/webwn?s=RFID\&sub=Search+WordNet\&o2=\&o0=1\&o8=1\&o1=1\&o7=\&o5=\&o9=\&o6=\&o3=\&o4=\&h=0000000}. Accessed 23 July 2021.

\item Word: exonerate

Definition:  ex·on·er·ate | iɡˈzänəˌrāt | verb [with object ] 1 (especially of an official body) absolve (someone) from blame for a fault or wrongdoing, especially after due consideration of the case: they should exonerate these men from this crime

Ref: WordNet Search - 3.1. \url{http://wordnetweb.princeton.edu/perl/webwn?s=exonerate\&sub=Search+WordNet\&o2=\&o0=1\&o8=1\&o1=1\&o7=\&o5=\&o9=\&o6=\&o3=\&o4=\&h=0000000}. Accessed 23 July 2021.

\item Word: discourse

 Definition: noun | ˈdisˌkôrs | written or spoken communication or debate: an imagined discourse between two people
traveling in France.

Ref: WordNet Search - 3.1. \url{http://wordnetweb.princeton.edu/perl/webwn?s=discourse\&sub=Search+WordNet\&o2=\&o0=1\&o8=1\&o1=1\&o7=\&o5=\&o9=\&o6=\&o3=\&o4=\&h=0000000}. Accessed 23 July 2021.

\item Word: profilerate

Definition: (v)  \textbf{profilerate} increase rapidly in numbers; multiply.

Ref: WordNet Search - 3.1.
\url{http://wordnetweb.princeton.edu/perl/webwn?s=profilerate\&sub=Search+WordNet\&o2=\&o0=1\&o8=1\&o1=1\&o7=\&o5=\&o9=\&o6=\&o3=\&o4=\&h=0000000}. Accessed 23 July 2021.

\item Word: prodigious

Definition: pro·di·gious | prəˈdijəs | adjective, remarkably or impressively great in extent, size, or degree: the stove consumed a prodigious amount of fuel.

Ref: WordNet Search - 3.1.
\url{http://wordnetweb.princeton.edu/perl/webwn?s=prodigious\&sub=Search+WordNet\&o2=\&o0=1\&o8=1\&o1=1\&o7=\&o5=\&o9=\&o6=\&o3=\&o4=\&h=0000000}. Accessed 23 July 2021.

\item Word: clairvoyant

Definition: clair·voy·ant | ˌklerˈvoiənt | noun a person who claims to have a supernatural ability to perceive events in the future or beyond normal sensory contact

Ref: WordNet Search - 3.1.
\url{http://wordnetweb.princeton.edu/perl/webwn?s=clairvoyant\&sub=Search+WordNet\&o2=\&o0=1\&o8=1\&o1=1\&o7=\&o5=\&o9=\&o6=\&o3=\&o4=\&h=0000000}. Accessed 23 July 2021.
\end{enumerate}

\subsection{Ideas:}
\label{sec:org43de896}
\begin{enumerate}
\item The notion of privacy has become fuzzier at the same time as the
secrecy-enhancing technology of encryption has become widespread (Page 21)
\item His car had a black box-an EDR, that captured every detail about what was
going on before the crash (page 27)
\item Bits mediate our daily lives. It is almost as hard to avoid leaving digital
footprints as it is to avoid touching the ground when we walk
\item ``There is no patient confidentiality'' said Dr. Joseph Heyman. ``It's gone''
\end{enumerate}

\subsection{Journal Entry}
\label{sec:org2f096c7}
\begin{enumerate}
\item I agree with the notion that privacy has become fuzzier over time. As
encryption and security technologies are becoming more widespread, it seems
people are caring less and less about their privacy when really they should
be caring more. Companies give us a false sense of privacy, when really they
are breaching it more than ever.

The greatest example of this is google. When you open up www.google.com, you
can see multiple mentions of privacy. In reality, google is notorious for
using user information to target ads and search results. They have multiple
analytics and adsense services that companies can purchase.

\item I agree with the idea of having tracking devices in cars. If most people are
given the choice between getting an expensive ticket and facing criminal
charges or lying, most people will choose to lie. Devices like the EDR ensure
we can make a conclusion based on actual data rather than from the victims
point of view.

However, similar to the issue with cellphone locations  the key term is \emph{as
long as that information is kept private and in the hands of the government}.
Private companies shouldn't be able to get the locations of users and use it
when it isn't needed. Examples of this can be determining how to price
billboard advertising, requiring cars to be serviced ever \emph{x} miles.  There can
be certain exceptions (e.g. An insurance company trying to determine who is
at fault), but for the most part this information should be for the
government, and even then only for when the government absolutely requires it

\item Its true that now its extremely difficult to do anything without leaving
digital traces everywhere. I personally think this issue is linked to idea
\#1, people value convenience over privacy. Companies create a false sense of
privacy, and justify all the analytics with improved convenience.

Most people don't want to put effort into maintaining their privacy, or
resist changes to their workflow and life .

\item 
\end{enumerate}

\section{Ghosts in the Machine}
\label{sec:org7a0766e}
\subsection{Vocabulary}
\label{sec:org8b1113b}
\begin{enumerate}
\item Word: metadata

Definition: noun  Data that describes other data, as in describing the origin, structure, or characteristics of computer files,

Ref: WordNet Search - 3.1. \url{http://wordnetweb.princeton.edu/perl/webwn?s=metadata\&sub=Search+WordNet\&o2=\&o0=1\&o8=1\&o1=1\&o7=\&o5=\&o9=\&o6=\&o3=\&o4=\&h}=. Accessed 18 July 2021.

\item Word: open source software

Definition: Programs for which the source code is freely available and freely redistributable, with no commercial strings attached.

Ref: WordNet Search - 3.1. \url{http://wordnetweb.princeton.edu/perl/webwn?s=open-source-software\&sub=Search+WordNet\&o2=\&o0=1\&o8=1\&o1=1\&o7=\&o5=\&o9=\&o6=\&o3=\&o4=\&h}=. Accessed 18 July 2021.

\item Word: ascii

Definition: noun  (computer science) a code for information exchange between computers made by different companies; a string of 7 binary

Ref: WordNet Search - 3.1. \url{http://wordnetweb.princeton.edu/perl/webwn?s=ascii\&sub=Search+WordNet\&o2=\&o0=1\&o8=1\&o1=1\&o7=\&o5=\&o9=\&o6=\&o3=\&o4=\&h}=. Accessed 23 July 2021.

\item Word: steganography

Definition:  noun   The deliberate concealment of data within other data, as by embedding digitized text in a digitized image.

Ref: WordNet Search - 3.1. \url{http://wordnetweb.princeton.edu/perl/webwn?s=steganography\&sub=Search+WordNet\&o2=\&o0=1\&o8=1\&o1=1\&o7=\&o5=\&o9=\&o6=\&o3=\&o4=\&h=0000000}. Accessed 23 July 2021.

\item Word: blocks

 Definition: noun (Computer Science) a block is a segment of a large area
that can be used to assign data

Ref: WordNet Search - 3.1. \url{http://wordnetweb.princeton.edu/perl/webwn?s=block\&sub=Search+WordNet\&o2=\&o0=1\&o8=1\&o1=1\&o7=\&o5=\&o9=\&o6=\&o3=\&o4=\&h=0000000}. Accessed 23 July 2021.

\item Word: algorithm

Definition:  noun   A finite set of unambiguous instructions that, given some set of initial conditions, can be performed in a prescribed

Ref: WordNet Search - 3.1. \url{http://wordnetweb.princeton.edu/perl/webwn?s=algorithm\&sub=Search+WordNet\&o2=\&o0=1\&o8=1\&o1=1\&o7=\&o5=\&o9=\&o6=\&o3=\&o4=\&h=0000000}. Accessed 23 July 2021.

\item Word: pixel

Definition:  noun   One of the tiny dots that make up the representation of an image in a computer's memory.

Ref: WordNet Search - 3.1. \url{http://wordnetweb.princeton.edu/perl/webwn?s=pixel\&sub=Search+WordNet\&o2=\&o0=1\&o8=1\&o1=1\&o7=\&o5=\&o9=\&o6=\&o3=\&o4=\&h=0000000}. Accessed 23 July 2021.

\item Word: raster

Definition: (v)  noun  A bitmap image, consisting of a grid of pixels, stored as a sequence of lines.

Ref: WordNet Search - 3.1.
\end{enumerate}
\url{http://wordnetweb.princeton.edu/perl/webwn?s=raster\&sub=Search+WordNet\&o2=\&o0=1\&o8=1\&o1=1\&o7=\&o5=\&o9=\&o6=\&o3=\&o4=\&h=0000000}. Accessed 23 July 2021.

\begin{enumerate}
\item Word: render

Definition: transitive verb (Computers)  To convert (graphics) from a file into visual form, as on a video display.

Ref: WordNet Search - 3.1.
\end{enumerate}
\url{http://wordnetweb.princeton.edu/perl/webwn?s=render\&sub=Search+WordNet\&o2=\&o0=1\&o8=1\&o1=1\&o7=\&o5=\&o9=\&o6=\&o3=\&o4=\&h=0000000}. Accessed 23 July 2021.

\begin{enumerate}
\item Word: spam

Definition:  noun   Unsolicited e-mail, often of a commercial nature, sent indiscriminately to multiple mailing lists, individuals, or

Ref: WordNet Search - 3.1.
\end{enumerate}
\url{http://wordnetweb.princeton.edu/perl/webwn?s=spam\&sub=Search+WordNet\&o2=\&o0=1\&o8=1\&o1=1\&o7=\&o5=\&o9=\&o6=\&o3=\&o4=\&h=0000000}. Accessed 23 July 2021.

\subsection{Ideas:}
\label{sec:org77ad4fa}
\begin{enumerate}
\item Metadata can help or refute claims (Page 78)
\item With some clever programming, the process could be made unnoticeable, but so
far neither Microsoft nor Apple has made the necessary software investment (Page 102)
\item Free software is a matter of the users freedom to run, copy, distribute,
study, change, and improve the software (page 94)
\item If google holds your documents, they are accessible from anywhere the internet reaches
\end{enumerate}

\subsection{Journal Entry}
\label{sec:org6196b91}
\begin{enumerate}
\item As the text after that statement mentioned,  metadata can be easily altered
to match a criminals statement. Although most people won't know about
metadata and how to alter it, this makes it ineffective for most purposes.

Personally, I think we should remove most metadata. It ends up doing more
harm than good, especially in the case of images, where metadata can help
trace the location where the image was taken. In the case of documents, it
provides easily forge-able data that serves no purpose.

\item In my opinion, there isn't a need to zero all abandoned blocks by default.
This is for several reasons. Firstly, the majority of people have no
intention to sell their storage. The majority of laptops nowadays have
storage soldered down, which would make those drives impossible to resell.
Secondly, those who care about their privacy will likely have other methods
to zero their storage anyways.

Additionally, making this zero-ing behavior the default will bring
performance implications as mentioned in the textbook. It's a slow process
writing to all of the abandoned blocks. The author mentions this issue could
be solved ``with some clever programming,'' but programming isn't magic, in the
end to properly erase these blocks you will have to write to then. A better
solution would be to incorporate this behavior into the filesystem itself,
prioritizing writing to abandoned blocks before free ones. This system
increases drive longevity, reduces the chance of abandoned blocks, and is
already incorporated in many filesystems today

\item I support the Free and Open Source movement. Open source software gives power
to the users. With the open source model come better security, more
features, and more support for users. Compared to commercial (paid) software,
users can not only identify and report bugs, but also fix those bugs
themselves. Since the source code for open source applications is available,
you can clone the repository, change the insecure code, and submit your
changes.  On the other hand, with commercial software, you will have to wait
until the company updates the application.

Additionally, you can add features to open source software. If Microsoft Word is
missing a feature, you can only request Microsoft to add it. However, if
there is an open source editor missing a feature, you can simply fork it and
add the feature yourself. Likewise, most open source editors will never die,
since there is always a community of people working on it, whereas most
commercial software is dependant on the future of the company producing it.

However, there are also drawbacks with open source software. Since there is
no financial backing for most projects, maintainers don't have an incentive
to keep working. Similarly, since coders aren't paid the quality work may not
be as good as that of a commercial project.

\item 
\end{enumerate}
\end{document}
